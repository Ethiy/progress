\documentclass[../main.tex]{subfile}

\begin{document}
    \section{Experiments}
    
    \subsection{Problème formulation}

    Based on the error grammar established earlier, a multitude of supervised classification problems can be studied:


    The features also can be modularly assembled into one big feature vector that is fed to the classifier.

    \begin{itemize}
        \item depth: 0
        \begin{itemize}
            \item Binary Classification: Qualifiable, Unqualifiable
        \end{itemize}
        \item depth: 1
        \begin{itemize}
            \item hierarchical: True
            \begin{itemize}
                \item LoD\@: 1
                \begin{itemize}
                    \item Multiclass\footnote{What is multiclass?} classification: No errors, Building errors
                \end{itemize}
                \item LoD\@: 2
                \begin{itemize}
                    \item Multiclass classification: No errors, Building errors, Facet errors.
                \end{itemize}
            \end{itemize}
            \item hierarchical: False
            \begin{itemize}
                \item LoD\@: 1
                \begin{itemize}
                    \item Multilabel\footnote{what is multilabel?} classification: No errors, Building errors
                \end{itemize}
                \item LoD\@: 2
                \begin{itemize}
                    \item Multilabel classification: No errors, Building errors, Facet errors.
                \end{itemize}
            \end{itemize}
        \end{itemize}
        \item depth: 2
        \begin{itemize}
            \item hierarchical: True
            \begin{itemize}
                \item LoD\@: 1
                \begin{itemize}
                    \item Multiclass classification: No errors, Building errors
                    \begin{itemize}
                        \item Error: No errors
                        \item Error: Building errors $\longrightarrow$ Multilabel classification
                    \end{itemize}
                \end{itemize}
                \item LoD\@: 2
                \begin{itemize}
                    \item Multiclass classification: No errors, Building errors, Facet errors.
                    \begin{itemize}
                        \item Error: No errors
                        \item Error: Building errors $\longrightarrow$ Multilabel classification
                        \item Error: Facet errors $\longrightarrow$ Multilabel classification
                    \end{itemize}
                \end{itemize}
            \end{itemize}
            \item hierarchical: False
            \begin{itemize}
                \item LoD\@: 1
                \begin{itemize}
                    \item Multilabel classification: 5 labels
                \end{itemize}
                \item LoD\@: 2
                \begin{itemize}
                    \item Multilabel classification: 9 labels
                \end{itemize}
            \end{itemize}
        \end{itemize}
    \end{itemize}

    \begin{itemize}
        \item geometric features
        \begin{itemize}
            \item geometric attributes
            \begin{itemize}
                \item $degree_{facets}$
                \item $areas_{facets}$
                \item $facets\_centroid\_distances_{facets}$
                \item $angles\_btw\_normals_{facets}$
                \item $nomals\_with\_relations_{facets}$
                \item $areas_{projected\_facets}$
                \item $length_{projected\_edges}$
            \end{itemize}
            \item statistic
            \begin{itemize}
                \item statistics:
                \begin{itemize}
                    \item max
                    \item min
                    \item median
                    \item mean
                    \item std deviation
                \end{itemize}
                \item histogram
                \begin{itemize}
                    \item resolution
                \end{itemize}
            \end{itemize}
        \end{itemize}
        \item altimetric_features
        \begin{itemize}
            \item statistics:
            \begin{itemize}
                \item max
                \item min
                \item median
                \item mean
                \item std deviation
            \end{itemize}
            \item histogram
            \begin{itemize}
                \item resolution
            \end{itemize}
        \end{itemize}
    \end{itemize}

    \subsection{Features visualization}

    I tried visualizing features by reducing their dimensions using PCA.\@ In Figure~\ref{fig::pca_viz}, we can see in snapshots what looks like structures in data.

    \thisfloatsetup{heightadjust=object}
    \setcounter{SubFigCounter}{1}
    \begin{figure}[H]
        \ffigbox{
            \ffigbox[\FBwidth]
            {
                \begin{subfloatrow}[3]
                    \captionsetup{labelformat=brace, justification=raggedright}
                    \ffigbox[\FBwidth]
                    {
                        \fbox{\includegraphics[width=.31\textwidth]{../images/vizualization/geom_feat_1.eps}}
                    }
                    {
                        \caption{We can see strips like structure of data.}\label{fig::geom_1}
                    }
                    \ffigbox[\FBwidth]
                    {
                        \fbox{\includegraphics[width=.31\textwidth]{../images/vizualization/geom_feat_2.eps}}
                    }
                    {
                        \caption{We cannot see much in this direction.}\label{fig::geom_2}
                    }
                    \ffigbox[\FBwidth]
                    {
                        \fbox{\includegraphics[width=.31\textwidth]{../images/vizualization/geom_feat_3.eps}}
                    }
                    {
                        \caption{We can observe some local structures but nothing global.}\label{fig::geom_3}
                    }
                \end{subfloatrow}
            }
            {
                \renewcommand\figurename{}
                \renewcommand{\thefigure}{(\roman{SubFigCounter})}

                \caption{Visualization of geometric features.}\label{fig::geom_viz}
                \refstepcounter{SubFigCounter}
            }
            \ffigbox[\FBwidth]
            {
                \begin{subfloatrow}[3]
                    \captionsetup{labelformat=brace, justification=raggedright}
                    \ffigbox[\FBwidth]
                    {
                        \fbox{\includegraphics[width=.31\textwidth]{../images/vizualization/alti_feat_1.eps}}
                    }
                    {
                        \caption{We can see how the classes are disposed in halo like structures.}\label{fig::alti_1}
                    }
                    \ffigbox[\FBwidth]
                    {
                        \fbox{\includegraphics[width=.31\textwidth]{../images/vizualization/alti_feat_2.eps}}
                    }
                    {
                        \caption{This take confirms the fact that `None' errors are in the center followed by `Facet' errors and `Building' erros at last.}\label{fig::alti_2}
                    }
                    \ffigbox[\FBwidth]
                    {
                        \fbox{\includegraphics[width=.31\textwidth]{../images/vizualization/alti_feat_3.eps}}
                    }
                    {
                        \caption{We can see clearly the halo like structures.}\label{fig::alti_3}
                    }
                \end{subfloatrow}
            }
            {
                \renewcommand\figurename{}
                \renewcommand{\thefigure}{(\roman{SubFigCounter})}

                \caption{Visualization of altimetric features.}\label{fig::alt_viz}
                \refstepcounter{SubFigCounter}
            }
            \ffigbox[\FBwidth]
            {
                \begin{subfloatrow}[3]
                    \captionsetup{labelformat=brace, justification=raggedright}
                    \ffigbox[\FBwidth]
                    {
                        \fbox{\includegraphics[width=.31\textwidth]{../images/vizualization/fused_feat_1.eps}}
                    }
                    {
                        \caption{We keep the same disposition of classes as in altimetric features.}\label{fig::fused_1}
                    }
                    \ffigbox[\FBwidth]
                    {
                        \fbox{\includegraphics[width=.31\textwidth]{../images/vizualization/fused_feat_2.eps}}
                    }
                    {
                        \caption{The same halos as in Figure~\ref{fig::alt_viz}~\ref{fig::alti_3} but more compact.}\label{fig::fused_2}
                    }
                    \ffigbox[\FBwidth]
                    {
                        \fbox{\includegraphics[width=.31\textwidth]{../images/vizualization/fused_feat_3.eps}}
                    }
                    {
                        \caption{The same forms as in Figure~\ref{fig::fused_viz}~\ref{fig::alti_2}.}\label{fig::fused_3}
                    }
                \end{subfloatrow}
            }
            {
                \renewcommand\figurename{}
                \renewcommand{\thefigure}{(\roman{SubFigCounter})}

                \caption{Visualization fused features.}\label{fig::fused_viz}
                \refstepcounter{SubFigCounter}
            }
        }
        {
            \caption{\label{fig::pca_viz} Visualization of \textit{PCA} reduced features.}
        }
    \end{figure}

    \subsection{Classification results}

    Two classification pipelines were tested:

    \begin{itemize}
        \item[(i).] Binary classification (depth: 0): the idea is to use geometric features to predict the unqualifiable builings which are to be discarded. They cannot be classified and can be determined by the operator. Table~\ref{tab::binary_rf_1000_4} shwos the results of a $10$-fold cross validation using Random Forests as a classifier with $1000$ trees and $4$ in deepth.

        \item[(ii).] Coarse Multi classification (depth: 1, hierarchical: True): classifies qualifiable buildings by the error classes: `None', `Building error', `Facet error'. We can observe the results of cross validated \textit{Random Forests} and \textit{SVM} in Figure~\ref{fig::class_viz}. We can also observe how facet relations play a big role in the classification in Figure~\ref{fig::feat_import}.
    \end{itemize}

    \begin{table}[H]
        \caption{\label{tab::binary_rf_1000_4}Binary classification results using a Random Forests (trees: $1000$, depth: $4$).}
        \begin{tabular}{c c c}
            \toprule
             & \textbf{Unqualifiable buildings} & \textbf{Qualifiable models} \\
            \midrule
            Occurence & $11.6$\% & $88.4$\% \\
            \midrule
            \midrule
            \multicolumn{3}{c}{\textbf{Cross validation results}}\\
            \midrule
            Metric & Max & Min \\
             \midrule
            train scores & $1.000$ & $0.9911$ \\
             \midrule
            fit time & $1.779$ secs & $1.693$ secs \\
             \midrule
            test scores & $1.000$ & $0.8775$\\
             \midrule
            score time & $0.2507$ secs & $0.2020$ secs\\
             \bottomrule
        \end{tabular}
    \end{table}

    \begin{figure}[H]
        \ffigbox{
            \ffigbox[\FBwidth]{
                \begin{subfloatrow}[1]
                    \captionsetup{labelformat=brace, justification=raggedright}
                        \fbox{\includegraphics[width=.75\textwidth]{../images/classification/rf_geom_depth_nbr50}}
                \end{subfloatrow}
            }
            {
                \caption*{(i). Visualization of \textit{Random Forest} results on geometric features.\ x-axis: depth and y-axis $\times 50$: number of trees.}
            }
            \ffigbox[\FBwidth]{
                \begin{subfloatrow}[1]
                    \captionsetup{labelformat=brace, justification=raggedright}
                        \fbox{\includegraphics[width=.75\textwidth]{../images/classification/rf_geom_altim_depth_nbr25}}
                \end{subfloatrow}
            }
            {
                \caption*{(ii). Visualization of \textit{Random Forest} results on fused features.\ x-axis: depth and y-axis $\times 25$: number of trees.}
            }
            \ffigbox[\FBwidth]{
                \begin{subfloatrow}[1]
                    \captionsetup{labelformat=brace, justification=raggedright}
                        \fbox{\includegraphics[width=.75\textwidth]{../images/classification/svm_C10pgo5_20_gamma_10pgo5_200}}
                \end{subfloatrow}
            }
            {
                \caption*{(iii). Visualization of \textit{rbf-kernel SVM} results. $10^{\frac{\text{x-axis}}{5}}$: $C$ and $10^{\frac{\text{y-axis}}{5}}$: $gamma$.}
            }
        }
        {
            \caption{\label{fig::class_viz} Visualization of Cross validated coarse classification scores.}
        }
    \end{figure}

    \begin{figure}[H]
        \ffigbox[\FBwidth]{
            \begin{subfloatrow}[1]
                \captionsetup{labelformat=brace, justification=raggedright}
                \fbox{\includegraphics[width=.75\textwidth]{../images/classification/feat_importance_rf}}
            \end{subfloatrow}
        }
        {
            \caption{\label{fig::feat_import} Pie chart visualizing feature importance.}
        }
    \end{figure}

    We need to classify now on subclasses and try to link the results back to real building in QGIS\@.
\end{document}
