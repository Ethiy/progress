\documentclass[a4paper, 11pt]{article}

\usepackage[utf8]{inputenc}
\usepackage[english]{babel}

\usepackage{hyperref}

\usepackage{graphicx}
\usepackage{float}

\usepackage{listings}

\usepackage{fullpage}

\begin{document}
	\begin{centering}
		\Large{\textbf{Progress Report}}\\
		\large{\today}
~\\
		Oussama ENNAFII\\
		Directors: Cl\'ement Mallet \& Florent Lafarge \\
		Advisor: Arnaud Le Bris \\

	\end{centering}


	\section{The Dataset:}
~\\

	We have decided earlier in June to work on Nantes and Dijon in addition to
	Elancourt. As I explained later, the $3DS$ data for these regions are not
	well formated so that I can retrieve the building entities. In order to
	establish first a whole pipeline. I will deal with those regions later using
	the $CityGML$ data. In consequence, I will continue to work on Elancourt
	only, for now.\\

	I have also relabelled the zone I had labelled before. That is due to the fact
	that the taxonomy has evolved during the annotation. I have used $QGIS$ to
	annotate buildings' projected facets based on errors they showed.\\

	The errors have been subdivided, based on the labelling, into three classes:
	\begin{itemize}
		\item[-] Unqualified Building Errors: Concerns the buildings that will not
		be taken into consideration,
		\item[-] Building Errors: encompasses the errors that affect the whole
		building (corrensponds roughly to $LoD0$ and $LoD1$ errors),
		\item[-] Facet Errors: concerns errors affecting only a facet.
	\end{itemize}



\section*{Attachments:}

\begin{itemize}
	\item[-] You can checkout the preprocessing code on
	\href{https://github.com/ethiy/proj.city}{Github}.
	\item[-] You can also check the feature extraction and classification code
	\href{https://github.com/ethiy/qualcity}{here}.
\end{itemize}

\end{document}
