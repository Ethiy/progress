\documentclass[a4paper, 11pt]{article}

\usepackage[utf8]{inputenc}
\usepackage[english]{babel}

\usepackage{hyperref}

\usepackage{graphicx}
\usepackage{float}

\usepackage{listings}

\usepackage{fullpage}

\begin{document}
	\begin{centering}
		\Large{\textbf{Progress Report}}\\
		\large{\today}
~\\
		Oussama ENNAFII\\
		Directors: Cl\'ement Mallet \& Florent Lafarge \\
		Advisor: Arnaud Le Bris \\

	\end{centering}





	\section{The Dataset:}
~\\

	We have decided earlier in June to work on Nantes and Dijon in addition to
	Elancourt. As I explained later, the \textit{3ds} data for these regions are
	not well formated so that I can retrieve the building entities. In order to
	establish first a whole pipeline. I will deal with those regions later using
	the \textit{cityGML} data. In consequence, I will continue to work on
	Elancourt only, for now.\\

	I have also relabelled the zone I had labelled before. That is due to the fact
	that the taxonomy has evolved during the annotation. I have used \textit{Qgis}
	to annotate buildings' projected facets based on errors they showed.


	\begin{itemize}
		\item I am concentrating on Elancourt right now:
		\begin{itemize}
			\item[-] The area encompasses an industrial area with shoe box buildings, a residential area with a lot of building similarities and a very sparse urban areas;
			\item[-] I have every data possible on the area: DSM, 3D building models and Orthoimages.
		\end{itemize}
	\end{itemize}

	\section*{Implementation:}
	Implementation status:
	\begin{itemize}
		\item Implemented:
			\begin{itemize}
				\item[-] Reader: 3DS and OFF,
				\item[-] Geomview viewer;
				\item[-] Automated testing;
				\item[-] Algorithms on bricks: area, contour length, affine transformations;
				\item[-] Project 3D models in 2D;
				\item[-] Occlusion management.
				\item[-] Save projections into georeferenced vectorial images using GDAL;
				\item[-] Rasterize projections for DSM comparison;
				\item[-] Save raster projections into georeferenced images using GDAL;
				\item[-] Stitch Meshes (in order to stitch rooftops to facets);
				\item[-] Read Scene tree from XML file.
				\item[-] Improved rasterization: with no artifacts, pixel size $= 6 cm $ $\Rightarrow$ all  Elancourt in one day.
			\end{itemize}
		\item To add:
			\begin{itemize}
				\item[-] Project 3D model on Camera.
		\end{itemize}
		\item To complete:
			\begin{itemize}
				\item[-] Improve logging.
			\end{itemize}
	\end{itemize}

	\begin{figure}[H]
		\caption{\label{diag::class} New changes to the program architecture.}
	\end{figure}

	\section*{Ideas:}
	I have identified two possible approaches in mind:
	\begin{itemize}
		\item[-] DSM and Z profil local comparison:
		\begin{itemize}
			\item[-] Simple Thresholding for inexistant buildings that are reconstructed with terrain height.
			\item[-] The difference still has structure - we can still see the facet borders - $\Rightarrow$ we can use the distribution of the difference (texture)  in order to determine if is noise or if the reconstruction missed something.
		\end{itemize}
		\item[-] Global building autoqualification:
		\begin{itemize}
			\item[-] Computed dual graph of each roof. Possibility of a bag of Features approach to get the similarity in neighborhoods.
		\end{itemize}

		\item[-] Orthoimage comparison:
		\begin{itemize}
			\item[-] Compute gradients "along" of building prints in order to detect any difference between inside or outside a facet/ building.
		\end{itemize}
	\end{itemize}

	\section*{Results:}
	\begin{figure}[H]
		\begin{center}
			\caption{\label{img::comparison} Example of building Z profile and its comparison with DSM: from left to right, up to bottom, the z profile, the  masked dsm , the laplacian of the difference, the orientation of the difference gradients, the amplitude of the difference gradient and the histogram corresponding to the orientations of the difference gradients.}
		\end{center}
	\end{figure}

	\begin{figure}[H]
		\begin{center}
			\caption{\label{img::comparison_bis} The same comparison applied for a building with chimney (not visible).}
		\end{center}
	\end{figure}

\section*{Attachments:}

\begin{itemize}
	\item[-] You can checkout the Code in \href{https://github.com/Ethiy/3DSceneModel}{Github}.
	\item[-] You can also check the \href{https://github.com/Ethiy/3DSceneModel/projects/1}{dev kaban}.
\end{itemize}

\end{document}
