\documentclass[9pt]{beamer}

\usepackage[utf8]{inputenc}

\RequirePackage{tikz}
\usetikzlibrary{arrows}
\graphicspath{{../images/theme/}}

\include{includes/IGN}

\usepackage[acronym]{glossaries}
\newacronym{acr::sample}{Accro}{Acronym}


\usepackage{amssymb}
\usepackage{amsmath}

\usepackage{mathtools}
\DeclarePairedDelimiter\ceil{\lceil}{\rceil}
\DeclarePairedDelimiter\norm{\vert\vert}{\vert\vert}
\DeclarePairedDelimiter\abs{\vert}{\vert}

\usepackage{hyperref}

\usepackage{color}


\title{Title}
\subtitle{SubTitle:}
\author{Author McAuthor}
\institute{IGN | K316}
\date{\today}


\AtBeginSubsection[]
{
	\begin{frame}<beamer>
		\frametitle{Presentation Layout}
		\tableofcontents[
		currentsection,
		sectionstyle=show/shaded,
		subsectionstyle=show/shaded/hide
		]
	\end{frame}
}

\AtBeginSection[]
{
	\begin{frame}<beamer>
		\frametitle{Presentation Layout}
		\tableofcontents[
		currentsection,
		sectionstyle=show/shaded,
		subsectionstyle=hide
		]
	\end{frame}
}

\begin{document}
	% Cover layout
	\usebackgroundtemplate{
		\begin{tikzpicture}
			\draw (0,0.5) node[right] { \includegraphics[width=12.5cm]{fondClair.png} };
			\draw (0,5) node[right] { \includegraphics[height=1.5cm]{LOGO_IGN.png} };
			\draw (1.5,4.95) node[right] { \includegraphics[width=3cm]{fondD.jpg} };
		\end{tikzpicture}   
	}
	
	\begin{frame}[plain,c]
		\begin{columns}
			\begin{column}{8cm}
				\begin{center}
					\vspace{2cm}
					\titlepage
				\end{center}
			\end{column}
			\begin{column}{1cm}
			\end{column}
		\end{columns}
	\end{frame}
	
	% Background layout
	\usebackgroundtemplate{
		\begin{tikzpicture}[scale=0.503]
			\filldraw[color=IGNGris] (0,0) rectangle(2.62,0.54);
			\filldraw[color=IGNGris] (4.77,0) rectangle(2.62+4.77,0.54);
			
			\filldraw[color=IGNVert] (9.50+0.27,0) -- (9.50,0.54)-- (9.50+7.41-0.27,0.54)-- (9.50+7.41,0)--cycle;
			
			\filldraw[color=IGNGris] (19.50,0) rectangle(2.62+19.50,0.54);
			\filldraw[color=IGNGris] (23.11,0.54)--(23.11+0.54,0)--(2.3+23.11,0)--(2.3+23.11,0.54)--cycle;;
		\end{tikzpicture}   
	}
		
	% Plan layout
	\setbeamercolor{section in sidebar}{fg=white } %LIGNE NECESSAIRE POUR EFFACER LE PLAN DE LA SIDEBAR (PAS MIEUX)
	\setbeamercolor{subsection in sidebar}{fg=white }%LIGNE NECESSAIRE POUR EFFACER LE PLAN DE LA SIDEBAR (PAS MIEUX)
	\setbeamercolor{section in sidebar shaded}{fg=white }%LIGNE NECESSAIRE POUR EFFACER LE PLAN DE LA SIDEBAR (PAS MIEUX)
	\setbeamercolor{subsection in sidebar shaded}{fg=white }%LIGNE NECESSAIRE POUR EFFACER LE PLAN DE LA SIDEBAR (PAS MIEUX)
	\begin{frame}
		\tableofcontents
	\end{frame}
	\setbeamercolor{section in sidebar}{fg=IGNGris}%LIGNE NECESSAIRE POUR AFFICHER LE PLAN DE LA SIDEBAR (PAS MIEUX)
	\setbeamercolor{subsection in sidebar}{fg=IGNGris}%LIGNE NECESSAIRE POUR AFFICHER LE PLAN DE LA SIDEBAR (PAS MIEUX)
	\setbeamercolor{section in sidebar shaded}{fg=IGNVert}%LIGNE NECESSAIRE POUR AFFICHER LE PLAN DE LA SIDEBAR (PAS MIEUX)
	\setbeamercolor{subsection in sidebar shaded}{fg=IGNVert}%LIGNE NECESSAIRE POUR AFFICHER LE PLAN DE LA SIDEBAR (PAS MIEUX)



	\section*{Introduction}
	\begin{frame}{Introduction}
		\begin{itemize}
			\item[-] Say something here.
		\end{itemize}
	\end{frame}
	
		
%%%%%%%%%%%%%%%%%%%%%%%%%%%%%%%%%%%%%%%%%%%%%%%%%%%%%%%%%%%%%%%%%%%%%%%%%%%%%%%%%%%%%%%%%%%%%%%%%%%%%%%%%
%%% SECTION
%%%%%%%%%%%%%%%%%%%%%%%%%%%%%%%%%%%%%%%%%%%%%%%%%%%%%%%%%%%%%%%%%%%%%%%%%%%%%%%%%%%%%%%%%%%%%%%%%%%%%%%%%
\section[Motivation]{Motivation:}

%%%%%%%%%%%%%%%%%%%%%%%%%%%%%%%%%%%%%%%%%%%%%%%%%%%%%%%%%%%%%%%%%%%%%%%%%%%%%%%%%%%%%%%%%%%%%%%%%%%%%%%%%
%%% SOUS SECTION
%%%%%%%%%%%%%%%%%%%%%%%%%%%%%%%%%%%%%%%%%%%%%%%%%%%%%%%%%%%%%%%%%%%%%%%%%%%%%%%%%%%%%%%%%%%%%%%%%%%%%%%%%
\subsection[Dimensionality]{Curse of dimensionality:}
\begin{frame}{Curse of dimensionality:}
	Some issues to take into account:
	\begin{itemize}
		\item[-] Samples cover exponenetially less volume while dimension grows: you need at least $ \ceil*{\frac{1}{\epsilon}^d}$ points to cover uniformly a hypercube, with a side length of $1$, at distance $\epsilon$.
		\item[-] Most intuition comming from low dimensions ( $ d < 2$ ) are not always valid: 
		\begin{itemize}
			\item[-] For instance, there is more volume in the corners of a hypercube rather than the center:
			$$\frac{V(\mathbb{B}_{2}(r))}{V(\mathbb{B}_{\infty}(r))}\xrightarrow[d \to \infty]{} 0$$
			\item[-] As the dimensionality increases, the less Euclidian distances are meaningful \cite{Domingos:2012:FUT:2347736.2347755}.
		\end{itemize}
	\end{itemize}
\end{frame}


%%%%%%%%%%%%%%%%%%%%%%%%%%%%%%%%%%%%%%%%%%%%%%%%%%%%%%%%%%%%%%%%%%%%%%%%%%%%%%%%%%%%%%%%%%%%%%%%%%%%%%%%%

%%%%%%%%%%%%%%%%%%%%%%%%%%%%%%%%%%%%%%%%%%%%%%%%%%%%%%%%%%%%%%%%%%%%%%%%%%%%%%%%%%%%%%%%%%%%%%%%%%%%%%%%%
%%% SECTION
%%%%%%%%%%%%%%%%%%%%%%%%%%%%%%%%%%%%%%%%%%%%%%%%%%%%%%%%%%%%%%%%%%%%%%%%%%%%%%%%%%%%%%%%%%%%%%%%%%%%%%%%%
\section[Scattering]{Scattering Transform:}

%%%%%%%%%%%%%%%%%%%%%%%%%%%%%%%%%%%%%%%%%%%%%%%%%%%%%%%%%%%%%%%%%%%%%%%%%%%%%%%%%%%%%%%%%%%%%%%%%%%%%%%%%

%%%%%%%%%%%%%%%%%%%%%%%%%%%%%%%%%%%%%%%%%%%%%%%%%%%%%%%%%%%%%%%%%%%%%%%%%%%%%%%%%%%%%%%%%%%%%%%%%%%%%%%%%
\subsection[Properties]{Mathematical Properties:}
\begin{frame}{Mathematical Properties:}
	\begin{block}{Scattering invariance properties:}
		\begin{itemize}
			\item[(i).] Let $G$ be a group. We note the action of $g \in G $ on an image $x(u)$: $x_g(u) = x(g^{-1}.u)$. We want the representation $\Phi$ to be invariant to the action of $G$ \cite{sifre2013rotation} iif :
			\begin{equation}
				\Phi(x_g)=\Phi(x)
			\end{equation}
			\item[(ii).] We want the representation $\Phi$ to be contractive \cite{anden2014deep} in order to preserve similarity:
			\begin{equation}
				\norm*{\Phi(x)-\Phi(y)} \leq \norm*{x-y}
			\end{equation}
			\item[(iii).] We want the representation to be stable to small deformations \cite{mallat2012group}: for any diffeomorphism $\tau$ such that $ \norm*{\nabla \tau} < 1$, 
			\begin{equation}
				\norm*{\Phi(x) - \Phi(x_{\tau})} \leq C.\norm*{\nabla \tau}.\norm*{x}
			\end{equation}
		\end{itemize}
	\end{block}
\end{frame}

%%%%%%%%%%%%%%%%%%%%%%%%%%%%%%%%%%%%%%%%%%%%%%%%%%%%%%%%%%%%%%%%%%%%%%%%%%%%%%%%%%%%%%%%%%%%%%%%%%%%%%%%%
%%% SECTION
%%%%%%%%%%%%%%%%%%%%%%%%%%%%%%%%%%%%%%%%%%%%%%%%%%%%%%%%%%%%%%%%%%%%%%%%%%%%%%%%%%%%%%%%%%%%%%%%%%%%%%%%%
\section[Issues]{Practical Issues:}

%%%%%%%%%%%%%%%%%%%%%%%%%%%%%%%%%%%%%%%%%%%%%%%%%%%%%%%%%%%%%%%%%%%%%%%%%%%%%%%%%%%%%%%%%%%%%%%%%%%%%%%%%
\subsection[Numerical]{Numerical considerations:}

\begin{frame}{Numerical considerations:}
	\begin{itemize}
		\item[-] We scatter only along frequency decreasing paths $0<j_k\leq j_{k+1}<J$. Other coefficients are close to zero.
		\item[-] Most of the energy of the transform is concentrated in the first layers:
		\item[-] Implementations:
		\begin{itemize}
			\item[-] \href{https://github.com/scatnet/scatnet}{ScatNet available on GitHub} (Apache 2.0).
		\end{itemize}
	\end{itemize}
\end{frame}

\subsection[Textures]{Texture discrimination:}

\subsection[Comparison]{Comparison with :}

	
	
	\section*{References}
	\begin{frame}[allowframebreaks]{References}
		\bibliographystyle{alpha}
		\bibliography{references.bib}
	\end{frame}
	
	
	\usebackgroundtemplate{
		\begin{tikzpicture}
		\draw (0,0.5) node[right] { \includegraphics[width=12.5cm]{fondClair.png} };
		\draw (0,5) node[right] { \includegraphics[height=1.5cm]{LOGO_IGN.png} };
		\draw (1.5,4.95) node[right] { \includegraphics[width=3cm]{fondD.jpg} };
		\end{tikzpicture}   
	}
	\begin{frame}[plain,c]
		\vspace{3cm}
		\begin{tikzpicture}
		\draw (0,0)node{};
		\draw (4.5,1) node[color =IGNGris, inner sep=0.5em, minimum size=0.5em, text centered,font=\LARGE] { Thanks for your attention, }; 
		\draw (4.5,0) node[color =IGNGris, inner sep=0.5em, minimum size=0.5em, text centered,font=\LARGE] { I am ready for your questions! }; 
		\draw (4.5,-2) node[color =IGNVert, inner sep=0.5em, minimum size=0.5em, text centered,font=\LARGE] { oussama.ennafii@ign.fr }; 
		\end{tikzpicture}
	\end{frame}
	
	
\end{document}
