\documentclass[a4paper, 11pt]{article}

\usepackage[utf8]{inputenc}
\usepackage[english]{babel}

\usepackage{hyperref}

\usepackage{graphicx}
\usepackage{float}

\usepackage{fullpage}

\begin{document}
	\begin{centering}
		\large\textbf{Meeting Summary: 21/12/2016}\\
		~\\
		Oussama ENNAFII:
		\normalsize MATIS | TITANE \\
		Directors: Cl\'ement Mallet \& Florent Lafarge \\
	\end{centering}
	
	~\\~\\
	This summary reports the main resolutions taken during the progress meeting that took place in office \textbf{\#K316} in \textbf{12/21/2016}.
	\section*{Logistics:}
	Every one was present for this meeting.
	\begin{itemize}
		\item[-] Oussama ENNAFII,
		\item[-] Cl\'ement Mallet: IGN thesis co-director,
		\item[-] Florent Lafarge: TITANE thesis co-director ( joining from Skype).
	\end{itemize}
	
	\section*{Agenda:}
	
	We discussed: 
	\begin{itemize}
		\item[(i)] Data availability,
		\item[(ii)] Feature descriptors,
		\item[(iii)] Implementation issues.
	\end{itemize}
	~\\
	
	Regarding data, we discussed the need for oriented images and the issues that would arise. We discussed Lidar availability and relevance for our problem. The feature description issue was a more intuition based and had no quantitative analysis. It was crystallized that it was adamant to bring a closure to the implementation issues the sooner possible, in order to advance regarding the feature crafting part.
	
	\section*{Decisions:}
	
	Based on the discussion, It was decided to act on the following plan:
	\begin{itemize}
		\item[(i)] Ask Yannick Couturier for any DSM that he can get his hands on while waiting for the official repository,
		\item[(ii)] Ask Sebastien for contacts regarding the oriented aerial images,
		\item[(iii)] Explain the library structure using simple diagrams,
		\item[(iv)] Compute some attributes - contour length, slope, air or projections for instance - in order to check data,
		\item[(v)] Visualize these attributes in 2D and in 3D.
	\end{itemize}

\end{document}
